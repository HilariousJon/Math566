% quiver style
\usepackage{tikz-cd}
% `calc` is necessary to draw curved arrows.
\usetikzlibrary{calc}
% `pathmorphing` is necessary to draw squiggly arrows.
\usetikzlibrary{decorations.pathmorphing}

% A TikZ style for curved arrows of a fixed height, due to AndréC.
\tikzset{curve/.style={settings={#1},to path={(\tikztostart)
					.. controls ($(\tikztostart)!\pv{pos}!(\tikztotarget)!\pv{height}!270:(\tikztotarget)$)
					and ($(\tikztostart)!1-\pv{pos}!(\tikztotarget)!\pv{height}!270:(\tikztotarget)$)
					.. (\tikztotarget)\tikztonodes}},
	settings/.code={\tikzset{quiver/.cd,#1}
			\def\pv##1{\pgfkeysvalueof{/tikz/quiver/##1}}},
	quiver/.cd,pos/.initial=0.35,height/.initial=0}

% TikZ arrowhead/tail styles.
\tikzset{tail reversed/.code={\pgfsetarrowsstart{tikzcd to}}}
\tikzset{2tail/.code={\pgfsetarrowsstart{Implies[reversed]}}}
\tikzset{2tail reversed/.code={\pgfsetarrowsstart{Implies}}}
% TikZ arrow styles.
\tikzset{no body/.style={/tikz/dash pattern=on 0 off 1mm}}

% useful macro for class
\newcommand{\at}[3]{\left.#1\right\vert_{#2}^{#3}}
\newcommand\quotient[2]{
	\mathchoice
	{% \displaystyle
		\text{\raise1ex\hbox{$#1$}\Big/\lower1ex\hbox{$#2$}}%
	}
	{% \textstyle
		#1\,/\,#2
	}
	{% \scriptstyle
		#1\,/\,#2
	}
	{% \scriptscriptstyle
		#1\,/\,#2
	}
}

\newcommand{\True}{\textsf{True}}
\newcommand{\T}{\textsf{T}}
\newcommand{\False}{\textsf{False}}
\newcommand{\F}{\textsf{F}}
\newcommand{\act}{\rotatebox[origin=c]{-180}{\(\,\circlearrowright\,\)}}

\usepackage{physics}
\usepackage{complexity}

\DeclareMathOperator{\id}{id}
\DeclareMathOperator{\supp}{supp}
\DeclareMathOperator{\vol}{vol}
\DeclareMathOperator{\dist}{dist}
\DeclareMathOperator{\diam}{diam}
\DeclareMathOperator{\im}{Im}
\DeclareMathOperator{\sgn}{sgn}
\DeclareMathOperator{\Int}{Int}
\DeclareMathOperator{\diag}{diag}
\DeclareMathOperator{\dom}{dom}
\DeclareMathOperator*{\argmax}{arg\,max}
\DeclareMathOperator*{\argmin}{arg\,min}
\DeclareMathOperator{\Var}{Var}
\DeclareMathOperator{\Cov}{Cov}
\DeclareMathOperator{\Corr}{Corr}
\DeclareMathOperator{\lcm}{lcm}
\DeclareMathOperator{\coker}{coker} % cokernel
\DeclareMathOperator{\Sym}{Sym} % symmetric group
\DeclareMathOperator{\Aut}{Aut} % automorphism group
\DeclareMathOperator{\End}{End} % endomorphism group
\DeclareMathOperator{\Hom}{Hom} % homomorphisms
\DeclareMathOperator{\Gal}{Gal} % galois group
\DeclareMathOperator{\ob}{Ob} % objects in category theory
\DeclareMathOperator{\vect}{Vect} % Vector objects in category theory
\DeclareMathOperator{\sets}{Sets} % set objects in category theory
\DeclareMathOperator{\GL}{GL} % general linear group
\DeclareMathOperator{\Id}{Id} % identity map
\DeclareMathOperator{\Fun}{Fun} % function group
\DeclareMathOperator{\reg}{reg} % regular representation
\DeclareMathOperator{\Span}{span} % span of vector space generator
\DeclareMathOperator{\Ind}{Ind} % induced representation
\DeclareMathOperator{\Gps}{Gps} % group category
\DeclareMathOperator{\Top}{Top} % topological space category

\renewcommand{\Re}{\operatorname{Re}}
\renewcommand{\Im}{\operatorname{Im}}

\let\implies\Rightarrow
\let\impliedby\Leftarrow
\let\iff\Leftrightarrow

\usepackage{stmaryrd} % for \lightning
\newcommand\conta{\scalebox{1.1}{\(\lightning\)}}

\usepackage{bm}
\usepackage{bbm}

% figure support
\usepackage{import}
\usepackage{xifthen}
\pdfminorversion=7
\usepackage{pdfpages}
\usepackage{transparent}
\newcommand{\incfig}[2][\columnwidth]{%
	\def\svgwidth{#1}
	\import{Figures/}{#2.pdf_tex}
}

% some efficient math alias
\newcommand{\CC}{\mathbb{C}} % Complex numbers
\newcommand{\FF}{\mathbb{F}} % Finite fields
\newcommand{\NN}{\mathbb{N}} % Natural numbers
\newcommand{\QQ}{\mathbb{Q}} % Rational numbers
\newcommand{\RR}{\mathbb{R}} % Real numbers
\newcommand{\ZZ}{\mathbb{Z}} % Integers
\newcommand{\KK}{\mathbb{K}} % general fields
\newcommand{\qo}{\quotient} % easier way to type quotient
\newcommand{\mb}{\mathbf}

% Automated Letter Loops (The "Magic" Part)
\foreach \x in {A,...,Z} {
    \expandafter\xdef\csname c\x\endcsname{\noexpand\mathcal{\x}} % mathcal 
    \expandafter\xdef\csname b\x\endcsname{\noexpand\mathbb{\x}} % mathbb 
    \expandafter\xdef\csname f\x\endcsname{\noexpand\mathfrak{\x}} % mathfrak 
    \expandafter\xdef\csname s\x\endcsname{\noexpand\mathscr{\x}} % mathscr 
}

% Paired Delimiters (Auto-sizing)
\DeclarePairedDelimiter{\ceil}{\lceil}{\rceil}  % Ceiling function
\DeclarePairedDelimiter{\floor}{\lfloor}{\rfloor} % Floor function
\DeclarePairedDelimiter{\angl}{\langle}{\rangle} % inner product or genrated idea
\DeclarePairedDelimiter{\br}{\llbracket}{\rrbracket} % french styled integer expression
\DeclarePairedDelimiter{\baceInner}{(}{)} % auto resized braces
\NewDocumentCommand{\bace}{s m}{
  \IfBooleanTF{#1}
    {\baceInner{#2}} % normal sized if \bace*
    {\baceInner*{#2}} % auto resized if \bace
}

% greek and msic symbols
\newcommand{\eps}{\varepsilon} % Better looking epsilon
\newcommand{\vp}{\varphi}      % Better looking phi
\newcommand{\lam}{\lambda}     % Shorthand for lambda
\newcommand{\half}{\frac{1}{2}} % Very common fraction
\newcommand{\inv}{^{-1}}       % Inverse operator shorthand
\newcommand{\rsa}{\rightsquigarrow} % some abution of arrow
% \newcommand{\iff}{\Longleftrightarrow} % Make 'iff' arrow longer if desired

% quotient 
% \newcommand{\quotient}[2]{{#1}/\mkern-2mu{#2}}
