\chapter{Ring Theory}

We've learnt about group theories which represents the symmetry for objects, which is kind of abstract. Rings are groups with extra structures, it is naturally more complicated, however it is closer to our intuition due to the same reason.

\section{Basic Definition}

\begin{definition}[\textbf{Ring}]
  A \underline{Ring} is a tuple \( (R,+,\cdot) \) being a set \( R \) endowed with \( 2 \) binary operations \( (+) \) and \( (\cdot) \), s.t.:
  \begin{enumerate}
    \item \( (R, +) \) is an \textbf{abelian} group, with identity element \( 0_R \) or \( 0 \).
    \item \( (\cdot) \) is associative, and has an identity element \( 1_R \) or \( 1 \).
    \item It satisfy distributivity:
      \begin{itemize}
        \item \( a \cdot (b+c) = (a\cdot b)+(a\cdot c), \; \forall \; a,b,c \in R\).
        \item \( (b+c)\cdot a = (b\cdot a) + (c\cdot a), \; \forall \; a,b,c \in R \).
      \end{itemize}
  \end{enumerate}
\end{definition}

\begin{notation}
  \leavevmode 
  \begin{enumerate}
    \item Usually write \( ab \) for \( a\cdot b \).
    \item If we don't use parentheses, the order of operations is First \( (\cdot) \) then \( (+) \).
    \item If \( (+), (\cdot) \) are understood, simply denote the ring by \( R \).
  \end{enumerate}
\end{notation}

\begin{remark}
  \leavevmode
  \begin{enumerate}
    \item As always, with identity elements \( 0_R, 1_R \) are unique.
    \item For every \( a\in R \), we have a unique inverse w.r.t \( (+) \), denoted by \( -a \).
    \item In general, don't require \( xy=yx \; \forall \; x,y \in R \), if this is the case, then \( R \) is a \underline{commutative ring}.
    \item Sometimes the definition of a ring does not require existence of \( 1_R \), such is defined as \underline{unitary ring}.
  \end{enumerate}
\end{remark}

\begin{eg}
  \leavevmode 
  \begin{enumerate}
    \item \( \ZZ, \RR, \QQ, \CC \) are rings w.r.t. \( (+), (\cdot) \).
    \item If \( n \in \ZZ_{>0} \), then \( \qo{\ZZ}{n\ZZ} \) carries two operations:
      \[
        \begin{aligned}
          [a]+[b] &\coloneqq  [a+b] \\ 
          [a]\cdot[b]&\coloneqq [ab]
        \end{aligned}
      \] 

      where \( [a]:=a+n\ZZ \), this is well-defined since operations holds regardless of the choice of representatives This is a ring with \( 0_{\qo{\ZZ}{n\ZZ}} = [0] \) and \( 1_{\qo{\ZZ}{n\ZZ}} = [1] \).
    \item Let \( R \) be any ring, then:
      \[
        M_{n}(R) \coloneqq  \{A = (a_{ij})_{1\leq i,j\leq n} \; | \; a_{ij} \in R \; \forall \; i,j\}
      \] 

      with ``usual'' addition and mult. for matrices:
      \[
        \begin{aligned}
          (a_{ij}) + (b_{ij}) &\coloneqq  (a_{ij}+ b_{ij}) \\ 
          (a_{ij})\cdot (b_{ij}) & \coloneqq  (c_{ij}) \rsa c_{ij} = \sum_{k=1}^n a_{ik} b_{kj}
        \end{aligned}
      \] 

      then \( (M_n(R), +, \cdot) \) is a ring with w.r.t. \( 1_{M_{n}(R)} = \begin{pmatrix}
        1_R & & 0_R \\ 
            & \ddots & \\ 
        0_R & & 1_R 
      \end{pmatrix} \).

      \begin{note}
        If \( n\geq 2 \), even if \( R \) is commutative, \( M_n(R) \) is not commutative in general.
      \end{note}

    \item Given a family \( (R_i)_{i\in I} \) of rings, where \( I \) may not be finite, define the following by \textbf{Cartesian Prod.}:
      \[
        \prod_{i\in I} R_i \coloneqq  \{(a_i)_{i\in I} \; | \; a_{i} \in R_i \; \forall \; i\}
      \] 

      define the operations \textbf{componentwise}:
      \[
        \begin{aligned}
          (a_i)_{i\in I} + (b_i)_{i\in I} &\coloneqq (a_i + b_i)_{i\in I} \\ 
          (a_i)_{i\in I} \cdot (b_i)_{i\in I} &\coloneqq  (a_i \cdot b_i)_{i\in I}
        \end{aligned}
      \] 

      with \( 0 = (0_{R_i})_{i\in I} \) and \( 1 = (1_{R_i})_{i\in I} \). If \( I = [n] \), simly write: \( R_1 \times \cdots \times R_n \).
  \end{enumerate}
\end{eg}

\begin{proposition}
  If \( R \) is a ring and \( a,b\in R \), then:
  \begin{enumerate}
    \item \( a\cdot 0_R = 0_R = 0_R \cdot a\).
    \item \( -(ab) = (-a)\cdot b = a\cdot (-b) \).
  \end{enumerate}

  The proof follows quickly from distributivity and the fact that \( (R,+) \) is an abelian group.
  \begin{note}
    If \( R \) is a set with \( 1 \) element \(\star \), then we can make it into a ring in a unique way, namely:
    \[
      0_R = 1_R = \star 
    \] 

    If \( R \) is a ring, then the following are equiv.:
    \begin{enumerate}
      \item \( \# R = 1 \). 
      \item \( R = \{0_R\} \).
      \item \( 1_R = 0_R \).
    \end{enumerate}

    proof is also trivial.
  \end{note}
\end{proposition}

\begin{definition}[\textbf{Ring Homomorphism}]
  Let \( R, S \) be two rings, the \underline{ring homomorphism} is a map \( f:R \to S \), such that:
  \begin{enumerate}
    \item \( f(a+b) = f(a)+f(b) \; \forall \; a,b \in R \).
    \item \( f(a\cdot b) = f(a) \cdot f(b) \; \forall \; a,b \in R\).
    \item \( f(1_R) = 1_S \).
  \end{enumerate}
\end{definition}

\begin{remark}
  \leavevmode 
  \begin{enumerate}
    \item If \( f:R\to S \) is a ring homo., then \( f:(R,+) \to  (S,+)\) is a group homomorphism, with \( f(0_R) = 0_S, \; f(a-b) = f(a)-f(b) \; \forall \; a,b \in R \).
    \item However, in def of ring hom. condition 3 \textbf{does not} implied by 1 and 2. 
  \end{enumerate}
\end{remark}

\begin{eg}
  If \( R = \{0_R\} \), then the only map \( f: R\to S \) that satisfies 1 and 2 in definition of ring homo. will satisfy:
  \[
    f(0_R) = 0_S
  \] 

  however, this does not satisfy condition 3 if \( S \ne \{0_S\} \).
\end{eg}

\begin{remark}
  In homework, we shall see if \( f: R\to S \), \( g: S\to T \) are ring homomorphisms, then \( g\circ f: R\to T \) is again a ring homomorphisms. In particular we have a \textbf{category} \underline{Rings}:
  \begin{itemize}
    \item Objects: rings.
    \item Morphisms: ring homomorphisms. 
    \item composition: usual function composition.
  \end{itemize}
\end{remark}
