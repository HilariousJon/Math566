\chapter{Lemma}

This chapter basically collect some linear algebra lemma that maybe helpful across the course.

\begin{lemma}
  Given \( A \) to be a matrix, who has the eigenvalues to be \( \lambda_1, \ldots, \lambda_p \), then the eigenvalues of the matrix \( A+c\Id \) where \( c \in \CC \) is \( \lambda_1 + c, \ldots, \lambda_p+c \).
\end{lemma}

\begin{lemma}
  \label{lem:nilpotent}
  Let \( A \) be as above, the eigenvalues of \( A^l \) are \( \lambda_1^l, \ldots, \lambda_p^l \).
\end{lemma}

\begin{lemma}
  \label{lem:poly-eigen}
  Let \( f \in \CC[t] \), then \( f(A) \) has eigenvalues being \( \{f(\lambda_i)\}_{i=1}^p \).
\end{lemma}

\begin{lemma}
  If \( A \) is a real and symmetric matrix, by spectral theorem for Hermitian product, we have \( \lambda_i \in \RR \).
\end{lemma}

\begin{lemma}
  \label{lem:match}
  Let \( \{\alpha_i\}_{i=1}^r \) and \( \{\beta_i\}_{i=1}^s \) be non-zero complex numbers, if for every \( l \geq 1 \), we have:
  \[
    \sum_{i=1}^{r}\alpha_i^l = \sum_{i=1}^s \beta_i^l 
  \] 

  then \( r=s \) and \(\{\alpha_i\} \) is permutation of \( \{\beta_i\} \).
\end{lemma}

\chapter{Walks in Graph}

\section{Graph Eigenvalues}

To better phrasing a graph, we first need to define multiset so that we can express not only simple graph but also general graph in a mathematically rigourous way.

\begin{definition}[\textbf{Multiset}]
  A \underline{multiset} \( M \) on a set \( S \) is an unordered collection of elements in \( S \), s.t.
  \begin{enumerate}
    \item \( \forall \; x \in M, x \in S \).
    \item The \( \# \) of times for \( x\in S \) to appear in \( M \), denoted as \( \mu_M(x) \), is \( \geq 0 \).
  \end{enumerate}
\end{definition}

\begin{eg}
  If \( \mu_M(x) = 0, 1 \; \forall x\), then \( M \) is a set.
\end{eg}

Note that two multiset \( M, M' \) are said to be equal if \( \forall x \in S, \mu_{M} (x) = \mu_{M'}(x)\).

\begin{notation}
  Let \( S \) be a finite set of size \( p \), then define:
  \[
    \binom{S}{k} = \{k-\text{subsets of } S\}
  \] 

  note that
  \[
    \abs{\binom{S}{k}} = \binom{\abs{S}}{k}
  \] 

  also define:
  \[
    \mchoo{S}{k} = \{k-\text{subsets of } S\}
  \] 

  note that:
  \[
    \abs{\mchoo{S}{k}} = \binom{p+k-1}{k}
  \] 

  this is the case: consider rephrasing the combinatorial problem as possible assigning of numbers for \( p \) numbers \( a_1, \ldots, a_p \) with \( a_1 + \ldots + a_p = k \) and \( a_i \in \br{0,k} \).
\end{notation}

Thus we can define the graph properly.

\begin{definition}[\textbf{Graph}]
  A \underline{finite graph} is a triple \( G=(V,E, \vp) \) with:
  \begin{itemize}
    \item \( V = \{v_1, \ldots, v_p\} \).
    \item \( E = \{e_1, \ldots, e_q\} \).
    \item \( \vp \) is a function \( E \to \mchoo{V}{2} \).
  \end{itemize}

  A finite simple graph is the same data with \( \vp: E \to \binom{V}{2} \)
\end{definition}

\begin{definition}[\textbf{Adjacency Matrix of Graph}]
  The \underline{adjacency matrix} of a graph \( G \), denoted as \( A(G) \), whose entries is defined by:
  \[
    a_{ij} = \abs{\vp^{-1}(\{v_i, v_j\})}
  \] 

  In particular it counts the number of edges between two vertices \( v_i \) and \( v_j \). Note that it is well-defined since if there is no edges between \( v_i \) and \( v_j \), then the preimage of \( \vp \) will be \( \emptyset \), thus \( a_{ij}=0 \).
\end{definition}

\begin{definition}[\textbf{Walk}]
  A \underline{walk} of length $k$ in a graph $G$ is a non-empty finite sequence of vertices and edges 
  \[ 
  W = v_0, e_1, v_1, e_2, v_2, \dots, e_k, v_k 
  \]
  such that for all $1 \le i \le k$, the edge $e_i$ has \textbf{endpoints} $v_{i-1}$ and $v_i$. 

  In a simple graph, where the edges are determined by their endpoints, a walk can be simplified to a sequence of vertices:
  \[ 
  W = (v_0, v_1, \dots, v_k) \quad \text{where } \{v_{i-1}, v_i\} \in E(G)
  \]
\end{definition}

\begin{note}
  \leavevmode 
  \begin{itemize}
    \item In a walk, the both the edges and vertices can appear \textbf{repeatedly}.
    \item If \( v_0 = v_k \), then such walk is called a \textbf{closed walk}.
  \end{itemize}
\end{note}

\begin{proposition}
  \label{prop:count-walk}
  For any integer \( l \geq 1 \), the \( (i,j) \) entry of \( (A(G))^l \), denoted as \( a_{ij} \), is equal to the \( \# \) of walks of length \( l \) in \( G \) starting from \( v_i \) to \( v_j \).
\end{proposition}

\begin{theorem}
  Let \( G \) be graph with \( A(G) \) possessing eigenvalues \( \lambda_1, \ldots, \lambda_p \), the \( \# \) \textbf{closed walks} of length \( l \) is:
  \[
    f_{G}(l) = \sum_{i=1}^p \lambda_i^l 
  \] 
\end{theorem}

The proof is straightforward combining the \textbf{Proposition} \ref{prop:count-walk} and \textbf{Lemma} \ref{lem:nilpotent}.
